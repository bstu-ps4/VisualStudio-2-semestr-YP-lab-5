\begin{enumerate}
    \item Вспомнить: указатели, ссылки; структуры данных (записи); динамические структуры: стек, дек; чтение данных из файла; функции: malloc, sizeof, free; new, delete.
    \item Разобраться с алгоритмом кодирования Шеннона-Фано и Хаффмана.
    \item Разобраться с принципами построения деревьев. Разработать подход построения бинарного дерева, который реализует соответствующий (вашему варианту) алгоритм кодирования.
    \item Написать программу:
    \begin{enumerate}
        \item Считать текст из файла. Это исходный текст, на основании которого будет происходить кодирование.
        \item Составить статистику по символам, встречающимся в тексте, в отсортированном виде: <Символ> <Частота> ... ...
        \item Реализовать необходимые типы/структуры для организации дерева. Разработать функции для работы с деревьями. Протестировать их работу, прежде чем приступать к реализации алгоритма кодирования.
        \item Реализовать алгоритм построения дерева.
        \item Отобразить (сохранить в файл) таблицу кодов для символов исходного текста.
    \end{enumerate}
\end{enumerate}

\newpage

\textbf{Вариант 3}

\lstinputlisting[
    name=text3.txt
]{../../CodeBlocks-2-semestr-YP-lab-5/files/text3.txt}

\newpage

\lstinputlisting[
    language=C,
    name=main.h
]{../../CodeBlocks-2-semestr-YP-lab-5/main.h}

\lstinputlisting[
    language=C,
    name=main.c
]{../../CodeBlocks-2-semestr-YP-lab-5/main.c}

\newpage

\lstinputlisting[
    name=Вывод в консоль
]{../includes_report/consoleOut.txt}

